\documentclass{article}

\usepackage{amsmath}
\usepackage{amssymb}
\usepackage{fancyhdr}

\title{DA1337/1338 Homework 6}
\author{tombergm}
\date{November 2023}

\fancyhead[L]{tombergm}
\pagestyle{fancy}

\begin{document}

\maketitle
\thispagestyle{fancy}

\renewcommand{\contentsname}{Innehållsförteckning}
\tableofcontents

\pagebreak
\section{Induktion}

\subsection{Problem 1}
Vi har ett påstående $P(n)$:
\[ P(n) \Leftrightarrow \sum_{i=1}^{n}{i^2} = \frac{n(n + 1)(2n + 1)}{6} \]
Basfall $P(1)$:
\[ \sum_{i=1}^{1}{i^2} = 1^2 = \frac{1(1 + 1)(2 \cdot 1 + 1)}{6} \Rightarrow P(1) \]
Induktionssteg; Antag $ P(n - 1) $:
\begin{align*}
\sum_{i=1}^{n}{i^2} & = \sum_{i=1}^{n - 1}{i^2} + n^2 \\
                    & = \frac{(n - 1)(n)(2n - 1)}{6} + n^2 \\
                    & = \frac{2n^3 - 3n^2 + n}{6} + \frac{6n^2}{6} \\
                    & = \frac{2n^3 + 3n^2 + n}{6} \\
                    & = \frac{n(n + 1)(2n + 1)}{6} \\
                    & \Rightarrow P(n)
\end{align*}
Vi har
\[ P(1) \land \left[P(n - 1) \Rightarrow P(n)\right] \]
V.S.B.

\subsection{Problem 2}
\[ P(n) \Leftrightarrow \sum_{j=1}^{n}{2j - 1} = n^2 \]
Basfall $ P(1) $:
\[ \sum_{j=1}^{1}{2j - 1} = (2 \cdot 1 - 1) = 1^2 \Rightarrow P(1) \]
Induktionssteg; Antag $ P(n - 1) $:
\[
\begin{aligned}
\sum_{j=1}^{n}{2j - 1} & = \sum_{j=1}^{n - 1}{2j - 1} + (2n - 1) \\
                       & = (n - 1)^2 + (2n - 1) \\
                       & = (n^2 - 2n + 1) + (2n - 1) \\
                       & = n^2 \\
\Rightarrow P(n)
\end{aligned}
\]
Därmed
\[ P(1) \land \left[P(n - 1) \Rightarrow P(n)\right] \]
V.S.B.

\section{Iterativ korrekthet}
Vi har en funktion $\operatorname{expIterative}(x, n) = f(x, n)$.
I början av loopen är
\[ res = res_0 = 1 = x^0 \]
I slutet av iteration $i$ är
\[ res = res_{i + 1} = res_i \cdot x = x^i \cdot x = x^{i + 1} \]
Den sista iterationen är iteration $n - 1$.
Efter loopen är
\[ res = res_{n - 1 + 1} = res_n = x^n \]
Returvärdet är
\[ f(x, n) = res_n = x^n \]
Antalet iterationer är $n$. Tidskomplexiteten är $ T(n) \in O(n)$.

\section{Rekursiv korrekthet}
Vi har en funktion $\operatorname{expRecursive}(x, n) = g(x, n)$.
\[
g(x, n) =
\begin{cases}
f(x, n) = x^n & 0 \leq n \leq 4 \\
g(x, \lfloor \frac{n}{2} \rfloor) \cdot g(x, \lfloor \frac{n + 1}{2} \rfloor) & n > 4
\end{cases}
\]
Låt påståendet $P(n)$:
\[ P(n) \Leftrightarrow g(x, n) = x^n \]
Vi har basfallen
\begin{equation}
\label{eq:gn}
g(x, n) = f(x, n) = x^n \Rightarrow P(n), n \in \left\{0, 1, 2, 3, 4\right\}
\end{equation}
Induktionsteg; Antag $ P(k), 2k > 4 $:
\begin{equation}
\label{eq:g2k}
g(x, 2k) = g(x, k) \cdot g(x, k) = x^k \cdot x^k = x^{2k} \Rightarrow P(2k)
\end{equation}
och $ P(k) \land P(k + 1), 2k + 1 > 4 $:
\begin{equation}
\label{eq:g2kp1}
g(x, 2k + 1) = g(x, k) \cdot g(x, k + 1) = x^{k} \cdot x^{k + 1} = x^{2k + 1} \Rightarrow P(2k + 1)
\end{equation}
Dessa två induktionssteg tillsammans med basfallen ger $ P(n) $ för $ n \in \mathbb{N} $, e.g.
\begin{gather*}
\begin{aligned}
P(2) \land P(3) & \Rightarrow &             & P(5)  && \text{enligt \eqref{eq:g2kp1}} \\
P(3) \land P(4) & \Rightarrow & P(6) \land  & P(7)  && \text{enligt \eqref{eq:g2k} och \eqref{eq:g2kp1}} \\
P(4) \land P(5) & \Rightarrow & P(8) \land  & P(9)  && \text{enligt \eqref{eq:g2k} och \eqref{eq:g2kp1}} \\
P(5) \land P(6) & \Rightarrow & P(10) \land & P(11) && \text{enligt \eqref{eq:g2k} och \eqref{eq:g2kp1}} \\
\dots
\end{aligned}
\end{gather*}

Tidskomplexiteten är av den rekursiva formen
\[ T(n) = aT\left(\frac{n}{b}\right) + f(n) \]
där
\begin{align*}
a    & = 2 \\
b    & = 2 \\
f(n) & \in \Theta(1) && \text{då $n \leq 4$ för alla
$\operatorname{expIterative}(x, n)$ i $\operatorname{expRecursive}$}
\end{align*}
Mästarsatsen ger
\[ T(n) \in \Theta(n^{log_b a}) = \Theta(n) \]

\end{document}
